Due to the absence of a tail this moment will be much smaller than on aircraft with tail.
Most of the wing surface area is behind the center of gravity and due to the wing sweep more surface will be at a longer distance from c.g. in $x$ direction.
This means that we can approximate the entire wing to be behind the c.g.
This aircraft also has a relatively small inertia around the $y$ axis; the aircraft will quickly turn into the wind.

Since we have modeled the lift in such a way that the pitching moment does not vary with angle of attack we can assume here that it is only proportional to the aircraft velocity and $\omega_y$, similarily as in section \ref{sec:restmomroll} but with constants instead of coefficients of lift and drag, giving us the following expression directly:
\begin{equation}
    M_y = -\omega_y \rho S \sqrt{v_x^2 + v_z^2} C_{D,\omega_y} \frac{x_{\omega_y}^2}{24}
    =
    -\omega_y \rho S C_{\omega_y} \sqrt{v_x^2 + v_z^2}
\end{equation}
where $C_{\omega_y}$ is some constant that needs to be approximated experimentally.
