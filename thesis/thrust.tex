Thrust will be modeled as
\begin{equation}
    T_i = K_T \omega_i^2
\end{equation}
where $K_T$ is the thrust coefficient of the propeller and $\omega_i$ the rotational speed of the propeller.

Behind the propellers an area of airflow is generated, called a wake.
Given a propeller diameter of $d$, pushing air through at a rate of $v_w$ the volume of air being moved by the propeller during time $t$ is:
\begin{equation}
	V_{air} = \int_0^t \frac{d}{2} \pi^2 v_w dt \,.
\end{equation}
The air has density $\rho$ converting the volume into mass:
\begin{equation}
	m_{air} = \rho \, V_{air}
\end{equation}
If we assume the air to be stationary in front of the propeller and the aircraft standing still the change in momentum in the air is:
\begin{equation}
	\Delta p = m_{air} V_{air} = \rho \frac{d \pi^2}{2} v_w^2 t
\end{equation}
Force due to change in momentum is:
\begin{equation}
	F t = \Delta p \Rightarrow F = \frac{\Delta p}{t}
\end{equation}
giving us:
\begin{equation}
	F = \rho  \frac{d \pi^2}{2} v_w^2 \, .
\end{equation}
By combining all parameters into one coefficient $K_v$, the following equation is obtained:
\begin{equation}
	F = K_v  v_w^2 \, .
	\label{propliftvel}
\end{equation}

If we instead look at the lift generated by the propeller blades we can find the thrust as a function of propeller rotation speed.
Assuming the aircraft has no speed and there is no wind we can model the lift from one propeller blade similarly as a wing, but with the angle of attack replaced with propeller blade pitch for clarity:
\begin{equation}
	L_i = \int_0^{\frac{d}{2}} \frac{\rho}{2} (\omega_i r)^2 C_L(pitch) dr
\end{equation}
If we also assume a constant pitch along the diameter of the propeller blade the above integral simplifies into:
\begin{equation}
	L_i = \frac{\rho d^3}{48} C_L(pitch) \omega_i^2
	\label{propliftrot}
\end{equation}
and thus with $n$ blades we find the total lift, i.e. thrust:
\begin{equation}
	T_i = L = \sum_{i \in [1,n]} L_i = n L_i
\end{equation}
By comparing equations \ref{propliftvel} and \ref{propliftrot} we conclude that:
\begin{equation} \begin{split}
	K_v v_w^2 =& n \frac{\rho d^3}{48} C_L(pitch) \omega_i^2 \Rightarrow \\
	& v_w = b_w \omega_i
\end{split}\end{equation}
and so the assumption that airflow in the wake is proportional to the propeller angular speed holds.

