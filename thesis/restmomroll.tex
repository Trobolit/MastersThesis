If we assume that $v_y$ is negligible then the Lift and Drag forces in body fram simplify into:
\begin{equation}\begin{split}
    L_i =& 
     S \frac{1}{2} \rho v^2 \left( \frac{C_L v_x + C_D v_z}{\sqrt{v_x^2 + v_z^2}} \right) = 
      S \frac{1}{2} \rho \sqrt{v_x^2 + v_z^2} \left(C_L v_x + C_D v_z \right)\\
     %
    D_i =& 
     S \frac{1}{2} \rho v^2 \left( \frac{-C_L v_z + C_D v_x}{ \sqrt{v_x^2 + v_z^2}} \right)
     = S \frac{1}{2} \rho \sqrt{v_x^2 + v_z^2} \left(-C_L v_z + C_D v_x\right)
     \label{liftdragbody}
\end{split}\end{equation}
If the plane is experiencing rolling, a rotation about the $x$-axis with a rotational speed of $\omega_x$ then a restoring moment will be created.
This is because the air is resisting the wings motion through the air.

The rotational speed will yield an increased local $v_z$, $v_{z,l}$, along the wing:
\begin{equation}
    v_{z, l} = v_z + \omega_x y
\end{equation}
where $y$ is the distance away from the body along the wing.
The change in velocity changes the angle of attack locally.
The local angle of attack is denoted $\alpha_l$.

Lets consider the full moment as the integral of the full moment across both wings.
This way the symmetrical lift forces will cancel and only the rolling moment will remain.
\begin{equation}\begin{split}
    M_x =& -\int_{-b/2}^{b/2} y dL_i(y) dy \\
    =& -\frac{\rho}{2}\int_{-b/2}^{b/2} y c(y) \sqrt{v_x^2 + v_{z,l}^2} \left(C_L v_x + C_D v_{z,l} \right) dy \\
\end{split}\end{equation}
If we make the crude assumption that $C_L = a * sin(2*\alpha)$, and the less crude asumption that $C_D = -b * cos(\alpha) + b$, it is still impossible to find the function $M_x(\omega_x)$ for the entire state space.
A term:
\begin{equation}
    \int_{-b/2}^{b/2} y \sqrt{(v_1 + \omega y)^2 + v_2^2} dy
\end{equation}
will remain, and one has to assume $\alpha \in [-\pi/2, \pi/2]$, $v_2 \neq 0$ and/or many of the variables always positive, to find the function without the integral.

A simulator can solve the integral numerically but in order to find smooth analytical expressions for the controller approximations have to be made.

It turns out that the four dimensional function for restoring moment due to roll as a function of roll rate $\omega_x$, $v_x$ and $v_z$ is actually quite linear with respect to $\omega_x$.
The remaining function can be closely approximated by a two dimensional polynomial in $v_x$ and $v_z$ allowing for a feedback linearized model to be implemented with very little error.
This can also help with computational efficiency should an estimator need to run onboard the aircraft.