Virginia Tech, a University in Virginia, USA, has developed a fixed wing drone called the EcoSoar; a flying wing.
It is an aircraft designed for fabrication, operation and maintenance in low-
resource environments.
The total cost of one EcoSoar is only around 350 USD when costs for all tools and materials have been accounted for and spread over ten aircraft. \cite{EcoSoarCost}

EcoSoar has been successfully flown in Malawi at the test corridor established by UNICEF in Kasungu, where a simulated payload of dried blood spot samples (used for HIV testing) was delivered 19 km from a remote health clinic to the Kasungu airport.
The aircraft is constructed of poster-board and 3D printed parts, making it easy to source in Malawi and low cost to repair if damaged.

Many flight operations occurring in Africa are converging on hybrid aircraft designs where the aircraft functions as a quadcopter in takeoff and landing but is capable of higher speed flight in a fixed wing configuration. \cite{swoopaero} \cite{wingcopter}

EcoSoar can currently be launched with a big bungee-cord or with a custom built launcher thus needing a field to both take off and land since it can only land on its belly similar to that of any normal aircraft.
Modifying the EcoSoar to have two engines, one on the front of each wing, instead of the current design with a single engine at the back, might allow it to hover, take off and land vertically.
Vertical take off and landing (VTOL) might eliminate the need for a launching system as well as the requirement of a big field for take-off and landing.
It also eleiminates the risk of high speed crashes at launch due to a badly calibrated aircraft.

Adding VTOL capacity to this drone would greatly increase its range of possible missions and thus help humanitarian efforts centered around EcoSoar in Malawi.

Hardware modifications along with a new control system onboard is necessary to achieve VTOL for the EcoSoar.

\subsection{EcoSoar hardware}
The EcoSoar frame is mostly built out of 3D printed parts for the body and spars.
The outer layer of the wings are made out of poster board and packing tape.
Small micro servos actuate on the Elevons for control.
The onboard electronics include a flight computer (Pixhawk 1), any compatible receiver, an ESC (Electronic Speed Controller) for the brushless motor powering the propeller, a telemetry module with antenna and a gps module. \cite{zack}