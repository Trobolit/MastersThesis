The aircraft considered here has no tail nor real body.
The only surfaces providing relevant forces due to sideslip are the winglets.

The winglets each have an area $S_W$ and are flat plate wings which will generate aerodynamic lift and drag forces due to $\beta$ providing an angle of attack in their reference frame.

$\beta$ can be, similarly to $\alpha$, be expressed in polar coordinates:
\begin{equation}\begin{split}
    v_x =& \, r \, cos(\beta) \\
    v_y =& \, r \, sin(\beta) \\
    r =& \sqrt{v_x^2 + v_y^2}
\end{split}\end{equation}



The aerodynamic forces, $L_\beta$ and $D_\beta$, from the winglets convert into body forces, directed as in figure \ref{airplane}, similarly to lift and drag from the main wings:
\begin{equation}\begin{split}
L_{W_i} &= cos(\beta) L_\beta +  sin(\beta)D_\beta \\
D_{W_i} & = -sin(\beta) L_\beta + cos(\beta) D_\beta
\end{split}\end{equation}
where
\begin{equation}\begin{split}
L_\beta &= Q C_{L_\beta} S_W \\
D_\beta & = Q C_{D_\beta} S_W
\end{split}\end{equation}
and $C_{k_\beta}$ are the lift and drag coefficients for each winglet, here assumed to be flat plates.
$v_z$ can be assumed to be negligible in the following equations since it will have no measurable effect on the winglets.
If we assume $v_z$ to be negligible then the equations simplify into
\begin{equation}\begin{split}
L_{W_i} &= 
    \frac{\rho v S_W}{2}  \left( C_{L_\beta} v_x + v_y C_{D_\beta} \right) \\
%
D_{W_i} &= 
    \frac{\rho v S_W}{2 }  \left(-v_y C_{L_\beta} + v_x C_{D_\beta} \right)
\end{split}\end{equation}

If the sideslip angle, $\beta$, is small (which implies $v_y$ small) the lift/drag coefficients can be approximated as 
\begin{equation}\begin{split}
    C_{L_\beta} \approx& \frac{d C_{L_\beta}}{d \beta} \big\vert_{\beta=0} \, \beta \\
    C_{D_\beta} \approx& 0
\end{split}\end{equation}
which results in:
\begin{equation}\begin{split}
L_{W_i} &\approx 
    \frac{\rho v S_W}{2} \frac{d C_{L_\beta}}{d \beta} \big\vert_{\beta=0} \, \beta \, v_x  \\
%
D_{W_i} &\approx 0
\end{split}\end{equation}
This gives us, in body frame, one restoring force from each winglet roughly proportional to the forward velocity squared of the aircraft and the sideslip angle..
The forces have the same magnitude and direction on both sides of the aircraft.



\subsubsection{Restoring moment}

$D_{W_i}$ are symmetrical on both sides and only create an acceleration in $y$.
They do not generate any moment.
$L_{W_i}$, however, act asymmetrically around c.g. and do generate a moment/torque (note that the force $L_{W_R}$ generates a negative torque around $z$ in our coordinate system):
\begin{equation}
    M_z = L_{W_L} x_W + L_{W_R} x_R = - \rho v S_W \frac{d C_{L_\beta}}{d \beta} \big\vert_{\beta=0} \, \beta \, v_x x_W
\end{equation}

\subsubsection{Non linear Forces and Moments}
If we do not make the assumption that $\beta$ is small we instead get the equations:
\begin{equation}\begin{split}
    M_z =& - \, x_W \rho v S_W  \left( C_{L_\beta}(\beta) v_x + v_y C_{D_\beta}(\beta) \right) \\
    D_{W} =& \rho v S_W \left(-v_y C_{L_\beta}(\beta) + v_x C_{D_\beta}(\beta) \right)
\end{split}\end{equation}
where $D_W = D_{W_L} + D_{W_R}$, and the coefficients of lift and drag are as shown in the figures above, but functions of $\beta$ instead of $\alpha$ as $\beta$ becomes the effective angle of attack in the winglets reference frame.



\subsubsection{Rolling moment due to sideslip}

The center of the winglet is slightly offset in the $z$-direction from the c.g. by distance $z_W$. 
Given the forces $L_{W_i}$ the rolling moment due to sideslip, $M_x$, is simply
\begin{equation}
    M_x = z_W (L_{W_L} + L_{W_R})
\end{equation}

